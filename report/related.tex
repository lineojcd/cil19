\section{Related Work}
\label{sec:related}

% traditional approaches (eg. edge detection)

% vgg, resnet
Deep learning solutions have been state of the art in image analysis tasks
since 2012 \cite{Kri12}. VGG nets \cite{Zis14} and ResNets \cite{He15} are
network architectures which benefit from multiple layers stacked successively
learning increasingly abstract features from input images. While ResNets can be
$8\times$ deeper than VGG nets, the key idea which counters problems such as
vanishing/exploding gradients associated with increasing depth is the ``skip
connection''

\citeauthor{Lon14} \cite{Lon14} developed fully convolutional networks (FCN) as
an extension to CNNs to perform semantic segmentation which have become the
de-facto standard in semantic egmentation. Combinations of FCNs with VGG nets
and ResNets have also been used to address segmentation in aerial images
\cite{Mar16} \cite{kai17} \cite{Azi18}.

U-Net is another network architecture built on top of FCNs. The main
characteristic of U-Net is the symmetric contracting (downsampling) and
expansive (upsampling) path composed with ``skip connections'' which
concatenate features from downsampling path to the equivalent upsampling
layers. A significant advantage of U-Net is less dependence on huge amounts of
data in contrast to other data hungry networks.  Though originally developed
for segmentation task in biomedical imaging, these have been used for automatic
road detection \cite{Zha17} \cite{Guo18}.

% bittel et al, coretin et. al. - address slightly different problems

